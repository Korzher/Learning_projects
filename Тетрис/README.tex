\documentclass[12pt, a4paper]{report}
\usepackage[utf8]{inputenc}
\usepackage[T2A]{fontenc}
\usepackage[english, russian]{babel}
\usepackage{verbatim} % для отображения кода

\title{Tetris Documentation}
\author{desmonsu}
\date{19.09.2025}

\begin{document}
\maketitle

\chapter{TETRIS}

Версия классического тетриса, написанная на языке C в рамках учебного проекта \texttt{C7\_brick\_game\_v1.0}. Оригинал написан Алексеем Пажитновым. Эта версия исполнена пиром desmonsu.

\section{Особенности}

\begin{enumerate}
\item Присутствует рандомная генерация всех тетрамино, представленных в оригинальной игре
\item Управление ходом игры реализовано с помощью конечного автомата
\item Серверная часть игры инкапсулирована от FSM и фронтенда
\item Реализована система подсчета очков и сохранения их в таблице рекордов
\item Реализована механика уровней в зависимости от набранных очков
\item Интерфейс соответствует оригинальной игре
\end{enumerate}

\section{Требования}

\begin{enumerate}
\item Компилятор gcc
\item Установленная библиотека ncurses
\item Система сборки Make
\end{enumerate}

\section{Установка}

\texttt{make install}

\section{Удаление}

\texttt{make uninstall}

\section{Запуск}

\begin{verbatim}
cd build/s21_tetris
./s21_tetris
\end{verbatim}

\section{Управление}

\begin{itemize}
\item Start - Ввод - начало игры, рестарт
\item Action - Пробел - поворот фигуры
\item Up - W - не используется в игре
\item Down - S - Падение фигуры (Hard drop)
\item Left - A - Сдвиг фигуры влево
\item Right - D - Сдвиг фигуры вправо
\item Terminate - Q - Выход из игры
\item Pause - P - Пауза
\end{itemize}

Кнопки букв регистронезависимы.

\section{Прочие команды Makefile}

\subsection{Запуск тестов}

\texttt{make test}

\subsection{Генерация gcov-отчета}

\texttt{make gcov\_report}

\subsection{Очистка артефактов сборки}

\texttt{make clean}

\subsection{Создание документации}

\texttt{make dvi}

\subsection{Создание дистрибутива}

\texttt{make dist}

\end{document}