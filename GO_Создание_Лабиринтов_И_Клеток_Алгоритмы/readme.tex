\documentclass[12pt, a4paper]{report}
\usepackage{fontspec}
\usepackage{polyglossia}
\setdefaultlanguage{russian}
\setotherlanguage{english}
\setmainfont{Liberation Serif}
\newfontfamily{\cyrillicfont}{Liberation Serif}
\newfontfamily{\cyrillicfonttt}{Liberation Mono}
\usepackage{verbatim}

\title{Maze Documentation}
\author{desmonsu, heraaero}
\date{18.01.2026}

\begin{document}
\maketitle

\chapter{Maze}

Генератор лабиринтов и пещер в рамках проекта школы21 \texttt{A1\_Maze\_Go}.

\section{Особенности}

\begin{enumerate}
\item Присутствует возможность подгрузки и сохранения кастомных лабиринтов и пещер
\item Рандомная генерация идеального лабиринта при помощи алгоритма Эллера
\item Генерация пещер с использованием клеточного автомата
\item Реализованы GUI-версия десктоп приложения и web-сервер
\item Имеется механика решения лабиринтов
\item Имеется возможность обучение ML-агента на выбранном пользователем лабиринте методом Q-обучения
\item Интерфейс соответствует оригинальной игре
\end{enumerate}

\section{Требования}

\begin{enumerate}
\item Компилятор go
\item Установленный фреймворк Wails и его зависимости (webview, webkit, nodejs, npm)
\item Система сборки Make
\end{enumerate}

\section{Установка}

\texttt{make install}

\section{Удаление}

\texttt{make uninstall}

\section{Запуск}

\begin{verbatim}
make run
make runconsole
make runweb
\end{verbatim}

\section{Ограничения}

\begin{itemize}
\item Размер поля для лабиринта или пещеры - 50х50
\item Числа в настройках не могут быть отрицательными
\item Числа для поиска пути не могут выходить за пределы лабиринта
\end{itemize}


\section{Прочие команды Makefile}

\subsection{Запуск тестов}

\texttt{make test}

\subsection{Очистка артефактов сборки}

\texttt{make clean}

\subsection{Создание документации}

\texttt{make dvi}

\subsection{Создание дистрибутива}

\texttt{make dist}

\end{document}